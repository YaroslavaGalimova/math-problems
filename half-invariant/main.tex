\documentclass[a4paper,12pt]{article}
\usepackage[utf8]{inputenc}
\usepackage[russian]{babel}
\usepackage{amsmath}
\usepackage{amssymb}
\usepackage{enumitem}
\usepackage{graphicx}
\usepackage{hyperref}
\usepackage[a4paper, top= 1cm, bottom=2cm, left=2cm, right=2cm]{geometry}

\hypersetup{
    colorlinks=true,
    linkcolor=blue,
    urlcolor=blue, 
}

\title{Полуинвариант}

\begin{document}
\maketitle
    \subsection*{Что это такое?} \textbf{\underline{Полуинвариантом}} называется величина, значение которой изменяется монотонно (не убывает или не возрастает) при заданных преобразованиях.
    \subsection*{Задачи}
        \begin{enumerate}
            \item На доске написаны три числа. Каждую минуту они изменяются по следующему правилу: если на данный момент на доске находятся числа x, y, z, то они меняются на $2x-2y + z - 10$, $2y - 2z + x + 21$, $2z - 2x + y - 12$ соответственно. Докажите, что когда-нибудь на доске появится отрицательное число.
            \item На доске написано натуральное число. Если на доске написано число x, то можно дописать на нее число $2x + 1$ или $3x$. В какой-то момент выяснилось, что на доске присутствует число 2024. Докажите, что оно там было с самого начала.
            \item Шеренга новобранцев стояла лицом к сержанту. По команде «налево» некоторые повернулись налево, а остальные — направо. Всегда ли сержант сможет встать в строй так, чтобы с обеих сторон от него оказалось поровну новобранцев, обращенных к нему лицом? Указание: поставить сержанта первым в строй и начать двигать вправо, исследуя, сколько человек обращены к нему лицом.
            \item На материке есть несколько стран, в каждой из которых правит либо партия правых, либо партия левых. Раз в месяц в одной из стран может поменяться власть. Это может произойти только в случае если в большинстве стран, граничащих с этой страной, правит другая партия. Докажите, что смены партий не могут продолжаться бесконечно. Указание: рассмотреть количество стран-соседей с разными партиями.
            \item В каждую клетку доски m на n записали по ненулевому целому числу. Разрешается поменять знак у всех чисел какой-то строки или какого-то столбца. Докажите, что можно такими действиями добиться неотрицательной суммы чисел в каждой строке и в каждом столбце.
            
        \end{enumerate}
\end{document}
