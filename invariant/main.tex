\documentclass[a4paper,12pt]{article}
\usepackage[utf8]{inputenc}
\usepackage[russian]{babel}
\usepackage{amsmath}
\usepackage{amssymb}
\usepackage{enumitem}
\usepackage{graphicx}
\usepackage{hyperref}
\usepackage[a4paper, top= 1cm, bottom=2cm, left=2cm, right=2cm]{geometry}

\hypersetup{
    colorlinks=true,
    linkcolor=blue,
    urlcolor=blue, 
}

\title{Что-то меняется, а что-то не меняется}

\begin{document}
\maketitle
    \subsubsection*{}\href{https://kvant.mccme.ru/1976/02/poisk_invarianta.htm}{Статья на эту тему в журнале "Квант"}
    \subsection*{\underline{Инвариант}} Настало время познакомиться с ещё одним методом решения задач - поиском инварианта. Ключевая цель в таких задачах - найти какую-то величину, которая в процессе операций, описанных в задаче, будет оставаться \textbf{неизменной}, такая величина называется \textbf{инвариантом}. Это может быть как что-то оченвидно следующее из условий задачи, так и какая-нибудь величина, которую вы придумываете сами и на основе неё конструируете своё решение.
    \subsection*{Задачи}
    (Настоятельно рекомендую использовать уже полученные знания и посмотреть на чётность-нечётность объектов)
    \begin{enumerate}
        \item На доске написано 10 чисел - пять из них равняются 1, остальные пять равняют -1. Боб стирает любые два числа и записывает их произведение на доске, пока не останется одно число. Чему оно может быть равно?
        \item \begin{enumerate}
            \item 100 фишек выставлены в ряд, часть из них чёрного цвета, а часть белого. Причём первая фишка белая, а последняя обязательно чёрная. Разрешено менять местами любые две фишки одного цвета. Можно ли через несколько таких операций добиться того, чтобы ряд из 100 фишек располагался в обратном порядке? 
            \item 100 фишек выставлены в ряд. Разрешено менять местами две фишки, стоящие через одну фишку. Можно ли с помощью таких операций переставить все фишки в обратном порядке?
        \end{enumerate}
        \item В языке Древнего Племени алфавит состоит всего из двух букв: "М" и "О". Два слова являются синонимами, если одно из другого можно получить при помощи исключения или добавления буквосочетаний "МО" и "ООММ", повторяемых в любом порядке и любом количестве. Являются ли синонимами в языке Древнего Племени слова "ОММ" и "МОО"?
        \item Круг разделен на 6 секторов, в котором по часовой стрелке стоят числа 1,0,1,0,0,0. Можно прибавлять по единице к любым числам, стоящим в двух соседних секторах. Можно ли сделать все числа равными?
        \item На столе стоят 16 стаканов. Из них 15 стаканов стоят правильно, а один перевёрнут донышком вверх. Разрешается одновременно переворачивать любые четыре стакана. Можно ли, повторяя эту операцию, поставить все стаканы правильно?
    \end{enumerate}
\end{document}
