\documentclass[a4paper,12pt]{article}
\usepackage[utf8]{inputenc}
\usepackage[russian]{babel}
\usepackage{amsmath}
\usepackage{amssymb}
\usepackage{enumitem}
\usepackage[a4paper, top=1cm, bottom=2cm, left=2cm, right=2cm]{geometry}


\title{Давайте использовать остатки!}

\begin{document}

\maketitle

\section*{Задачи}

\begin{enumerate}[label=\textbf{\arabic*.}]
    \item Существует ли натуральное число, произведение цифр которого равно 390?
    \item \begin{enumerate}
        \item На столе стоят семь пустых коробок. За один ход можно положить по монете в любые четыре коробки. Можно ли за несколько ходов добиться того, чтобы сумма всех монет была нечётной? А чтобы в каждой из семи коробок было нечётное число монет?
        \item На столе стоят семь стаканов – все вверх дном. За один ход можно перевернуть любые четыре стакана. Можно ли за несколько ходов добиться того, чтобы все стаканы стояли правильно?
    \end{enumerate}
    \item Может ли n! оканчиваться ровно на пять нулей?
    \item \begin{enumerate}
        \item Найдите всевозможные остатки при делении на 9 куба натурального числа.
        \item Докажите, что $n^3+2$ не делится на 9 ни при каком натуральном n.
    \end{enumerate}
    \item На клетчатом листе закрасили 25 клеток. Может ли каждая из них иметь нечётное число закрашенных соседей?
    \item \begin{enumerate}
        \item Вспомните и сформулируйте признаки делимости на 5 и 9
        \item Найдите наименьшее число, кратное 45, десятичная запись которого состоит только из единиц и нулей.
    \end{enumerate}
    
\end{enumerate}

\end{document}
