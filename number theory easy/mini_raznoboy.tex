\documentclass[a4paper,12pt]{article}
\usepackage[utf8]{inputenc}
\usepackage[russian]{babel}
\usepackage{amsmath}
\usepackage{amssymb}
\usepackage{enumitem}
\usepackage[a4paper, top=1cm, bottom=2cm, left=2cm, right=2cm]{geometry}


\title{Совсем немного теории чисел}

\begin{document}

\maketitle

\section*{Введение}
Теория чисел - это раздел математики, который занимается изучением числа как самостоятельного объекта. Очень часто в задачах на теорию чисел нужно попробовать разложить на множители, посотреть на остатки при делении на какое-нибудь число, найти НОД / НОК или придумать другую красивую идею. Математика - это про творчество и нестандартный подход к решению задач, но перед этим нужно получить базу из классических подходов.

\section*{Задачи}

\begin{enumerate}[label=\textbf{\arabic*.}]
    \item  \underline{Упражнения:} Докажите, что сумма двух четных чисел всегда четная. Верно ли это для суммы двух нечётных чисел? А что будет, если умножить два чётных числа? Может ли число быть кратно 37, если оно представимо как произведение чисел, меньших 37?
    
    \item Исследуйте, при каких условиях произведение двух простых чисел остается простым.
    
    \item Может ли n! оканчиваться ровно на пять нулей?
    
    \item  \underline{Теорема Вильсона (половина)} $p$ - простое $ \Leftarrow{} (p - 1)! + 1 \mathop{\raisebox{-2pt}{\vdots}} p$

    \item Докажите признак делимости на 3. Число делится на 3, если сумма его цифр делится на 3.

    \item На доске написаны числа 2 и 3. Каждую минуту Вова перемножает все записанные на доске числа, прибавляет 1 и записывает на доске наибольший простой делитель получившегося числа. Появится ли когда-нибудь на доске число 5?
\end{enumerate}

\end{document}
