\documentclass[a4paper,12pt]{article}
\usepackage[utf8]{inputenc}
\usepackage[russian]{babel}
\usepackage{amsmath}
\usepackage{amssymb}
\usepackage{enumitem}
\usepackage{graphicx}
\usepackage{hyperref}
\usepackage[a4paper, top= 1cm, bottom=2cm, left=2cm, right=2cm]{geometry}

\hypersetup{
    colorlinks=true,
    linkcolor=blue,
    urlcolor=blue, 
}

\title{Разнобой}

\begin{document}
\maketitle
    \subsection*{Задачи}
    \begin{enumerate}
        \item Монету бросают трижды. Сколько разных последовательностей орлов и решек можно при этом получить?
        \item В футбольной команде (11 человек) нужно выбрать капитана и его заместителя. Сколькими способами это можно сделать?
        \item В гости пришло 10 гостей и каждый оставил в коридоре пару калош. Все пары калош имеют разные размеры. Гости начали расходиться по одному, одевая любую пару калош, в которые они могли влезть (т.е. каждый гость мог надеть пару калош, не меньшую, чем его собственные). В какой-то момент обнаружилось, что ни один из оставшихся гостей не может найти себе пару калош, чтобы уйти. Какое максимальное число гостей могло остаться?
        \item В клетчатом квадрате 5 × 5 каждую клетку покрасили в один изтрёхцветов:красный,синийилизелёный.Справаоткаждойстрокизаписалисуммарноеколичествосинихикрасныхклетоквэтойстрочке,аподкаждымстолбцомзаписали суммарное количество синих и зелёных клеток в этом столбце. Справа от таблицы оказались числа 1, 2, 3, 4, 5 в некотором порядке. Могли ли и под таблицей оказаться числа 1, 2, 3, 4, 5 в некотором порядке?
        \item Действительные числа $x_1, x_2, x_3, x_4$ таковы, что:
            \begin{equation}
            \begin{cases}
                    $x_1 + x_2 \geq 12$ \\
                    $x_1 + x_3 \geq 13$ \\
                    $x_1 + x_4 \geq 14$ \\
                    $x_3 + x_4 \geq 22$ \\
                    $x_2 + x_3 \geq 23$ \\
                    $x_2 + x_4 \geq 24$
            \end{cases}
            \end{equation}
        Какое наименьшее значение может принимать сумма $x_1+x_2+x_3+x_4$?
        \item Пусть \$ - наша новая операция. Число a\$b есть произведение b последовательных натуральных чисел, наименьшее из которых равно a (в частности, a\$1= a). Найдите все пары натуральных чисел a, b, для которых выполнено равенство a\$b=2(b\$a). 
        \item Назовём ход ладьи банальным, если она смещается на кратное трём число клеток. В противном случае назовём ход оригинальным. Может ли ладья обойти поле 9$\cdot$9, чередуя банальные и оригинальные ходы так, чтобы в каждой клетке ладья побывала ровно один раз? 
    \end{enumerate}
\end{document}
