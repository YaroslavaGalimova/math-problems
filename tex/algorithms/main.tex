\documentclass[a4paper,12pt]{article}
\usepackage[utf8]{inputenc}
\usepackage[russian]{babel}
\usepackage{amsmath}
\usepackage{amssymb}
\usepackage{enumitem}
\usepackage{graphicx}
\usepackage{hyperref}
\usepackage[a4paper, top= 1cm, bottom=2cm, left=2cm, right=2cm]{geometry}

\hypersetup{
    colorlinks=true,
    linkcolor=blue,
    urlcolor=blue, 
}

\title{Алгоритмы}
\date{}

\begin{document}
\maketitle
    \subsection*{О чём этот листочек?}
    Иногда в задачах полезно бывает не только придумать абстрактное решение, но и привести вполне конструктивный пример или алгоритм. Это помогает и в задачах, где напрямую об этом просят, и в задачах со сложными условиями, в которых легко запутаться. Гораздо легче решать задачу, если перед глазами есть хоть какой-нибудь пример.
    \subsection*{Задачи}
    \begin{enumerate}
    \item Совунья купила в магазине 3 котлеты, принесла их домой и решила их пожарить. К сожалению, оказалось, что на ее маленькой сковородке одновременно помещаются только 2 котлеты. Каждую котлету надо обжаривать с двух сторон. На обжарку котлеты с одной стороны необходимо потратить 2 минуты. Может ли Совунья пожарить котлеты меньше, чем за 8 минут?
    \item Петя загадал число от 1 до 64. Как за 6 вопросов, ответы на которые могут быть "да" или "нет" узнать, какое число он загадал?
    \item На столе лежат в ряд пять монет: средняя  — вверх орлом, а остальные  — вверх решкой. Разрешается одновременно перевернуть три рядом лежащие монеты. Можно ли при помощи нескольких таких переворачиваний все пять монет положить вверх орлом?
    \item Вариация задачи про волка, козу и капусту. Три миссионера и три каннибала должны пересечь реку в лодке, в которой могут поместиться только двое. Миссионеры должны соблюдать осторожность, чтобы каннибалы не получили на каком-либо берегу численное преимущество. Как переплыть реку?
    \item Имеются 12-литровый бочонок, наполненный квасом, и два пустых бочонка – в 5 и 8 л. Попробуйте, пользуясь этими бочонками:
    \begin{enumerate}
        \item разделить квас на две части – 3 и 9 л
        \item разделить квас на две равные части
    \end{enumerate}
    \item Имеются двое песочных часов – на 7 минут и на 11 минут. Яйцо варится 15 минут. Как отмерить это время при помощи имеющихся часов?
    \item Золотоискатель Джек добыл 9 кг золотого песка. Сможет ли он за три взвешивания отмерить 2 кг песка с помощью чашечных весов: 
    \begin{enumerate}
        \item с двумя гирями  — 200 г и 50 г
        \item с одной гирей 200 г?
    \end{enumerate}
    \item Лиса Алиса и Кот Базилио  — фальшивомонетчики. Базилио делает монеты тяжелее настоящих, а Алиса  — легче. У Буратино есть 15 одинаковых по внешнему виду монет, но какая-то одна  — фальшивая. Как двумя взвешиваниями на чашечных весах без гирь Буратино может определить, кто сделал фальшивую монету  — Кот Базилио или Лиса Алиса?
    \item Какое наименьшее число выстрелов в игре "Морской бой" на доске 7*7 нужно сделать, чтобы наверняка ранить четырехпалубный корабль (четырехпалубный корабль состоит из четырех клеток, расположенных в один ряд)?
    \item Петя и Вася играют в игру. Перед ними есть 7 предметов. Петя пишет название одного из них на бумажке, а затем прячет её. Далее оба,  начиная с Пети, задают второму вопрос: "Какой из этих двух предметов убрать?" и показывают на два оставшихся предмета. Вторй выбирает один из них, а второй оставляет на столе. В конце игры остаётся только один предмет. И о чудо! Его название написано на Петиной бумажке. Как ему это удалось?
    \item Неуловимый Джо никогда не проигрывает на рулетке больше четырех раз подряд и никогда не ставит больше 10 долларов. Как ему выиграть 1000 долларов? (В случае выигрыша на рулетке возвращается удвоенная ставка; вначале Джо имеет 100 долларов.)
    \item Ученики школы посещают кружки. Докажите, что можно несколько школьников принять в пионеры так, чтобы в каждом кружке был хотя бы один пионер и для любого пионера нашелся кружок, в котором он был бы единственным пионером.
        
    \end{enumerate}
\end{document}
