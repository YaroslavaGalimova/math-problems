\documentclass[a4paper,12pt]{article}
\usepackage[utf8]{inputenc}
\usepackage[russian]{babel}
\usepackage{amsmath}
\usepackage{amssymb}
\usepackage{enumitem}
\usepackage{graphicx}
\usepackage{hyperref}
\usepackage[a4paper, top= 1cm, bottom=2cm, left=2cm, right=2cm]{geometry}

\hypersetup{
    colorlinks=true,
    linkcolor=blue,
    urlcolor=blue, 
}

\title{Признаки делимости}

\begin{document}
\maketitle
    \subsection*{Введение}
    \begin{enumerate}
        \item Делимость на 2: Число делится на 2, если его последняя цифра четная (0, 2, 4, 6, 8).
        \item Делимость на 3: Число делится на 3, если сумма его цифр делится на 3.
        
        \item Делимость на 4: Число делится на 4, если последние две цифры образуют число, которое делится на 4.
        
        \item Делимость на 5: Число делится на 5, если его последняя цифра равна 0 или 5.
        
        \item Делимость на 6: Число делится на 6, если оно делится и на 2, и на 3.
        
        \item Делимость на 8: Число делится на 8, если последние три цифры образуют число, которое делится на 8.
        
        \item Делимость на 9: Число делится на 9, если сумма его цифр делится на 9.
        
        \item Делимость на 10: Число делится на 10, если его последняя цифра равна 0.
        
        \item Делимость на 11: Число делится на 11, если разность между суммой цифр на нечетных позициях и суммой цифр на четных позициях делится на 11.
    \end{enumerate}
    \subsection*{Задачи}
    \begin{enumerate}
    \item Какие цифры можно вставить вместо звездочки в число 1234$\star$6789   так, чтобы оно делилось на 3?
    \item Найдите самое маленькое натуральное число, которое делится на 2  , но не делится на 5  , а после переноса последней цифры в начало результат делится на 5  , но не делится на 2  .
    \item Можно ли в числе 123456789   переставить цифры так, чтобы оно делилось на каждую из своих цифр?
    \item Можно ли в числе 12345   переставить цифры так, чтобы оно стало квадратом?
    \item Является ли число 12345678901234   квадратом натурального числа?
    \item Найдите какое-нибудь 100-значное число без нулевых цифр, которое делится на сумму своих цифр. Подсказка: нужно придумать признак делимости на 125. Подсказка к подсказке: посмотрите на признак делимости на 8.
    \item На доске было написано число вида 77...77  . Петя стёр у этого числа последнюю цифру, полученное число умножил на 3   и к произведению прибавил стёртую цифру. С полученным числом он проделал такую же операцию, и так далее. В конце осталось однозначное число. Чему оно может быть равно?
    \item В Лёшиной книге рекордов Гиннесса написано, что наибольшее известное простое число равно $23021^{377} - 1$  . Напомним, что число называется простым, если оно имеет ровно два натуральных делителя: единицу и само это число. Нет ли ошибки в Лёшиной книге рекордов?
    \item На вопрос: “В каком году Вы родились?” Дмитрий Алексеевич не дал прямого ответа. Но сказал, что две последние цифры его года рождения такие же, как у произведения всех двузначных чисел, уменьшенного на 5  . Приглядевшись, вы заметили, что Дмитрию Алексеевичу меньше ста лет. В каком году родился Дмитрий Алексеевич?
    \item Назовём натуральное семизначное число удачным, если оно делится на произведение всех своих цифр. Существуют ли четыре последовательных удачных числа?
    \end{enumerate}
\end{document}
