\documentclass[a4paper,12pt]{article}
\usepackage[utf8]{inputenc}
\usepackage[russian]{babel}
\usepackage{amsmath}
\usepackage{amssymb}
\usepackage{enumitem}
\usepackage{graphicx}
\usepackage{hyperref}
\usepackage[a4paper, top= 1cm, bottom=2cm, left=2cm, right=2cm]{geometry}

\hypersetup{
    colorlinks=true,
    linkcolor=blue,
    urlcolor=blue, 
}

\title{Введение в алгебраические задачи}

\begin{document}
\maketitle
    \begin{enumerate}
    \subsection*{Околоалгебраические задачи}
    \item Число A положительно, В отрицательно, а C равно нулю. Каков знак числа AB+ AC+BC?
    \item На лужайке босоногих мальчиков столько же, сколько обутых девочек. Кого на лужайке больше — девочек или босоногих детей?
    \item Мальвина велела Буратино умножить число на 4 и к результату прибавить 15, а Буратино умножил число на 15 и потом прибавил 4, однако, ответ получился верный. Какое это было число?
    \subsection*{Формулы, сокращения и прочие полезные навыки}
    \item Раскройте скобки в выражении $(a+b+c)^2$.
    \item Известно, что  $a + b + c = 5$  и  $ab + bc + ac = 5$.  Чему может равняться  $a^2 + b^2 + c^2$?
    \item a, b, c – такие три числа, что  a + b + c = 0.  Доказать, что в этом случае справедливо соотношение  ab + ac + bc ≤ 0.
    \item Докажите равенство  $(a^2 + b^2)(u^2 + v^2) = (au + bv)^2 + (av - bu)^2$.
    \subsection*{Текстовые задачи}
    \item Среди людей, не говорящих по-английски, 4\%   говорят по-французски, а среди людей, не говорящих по-французски, 20\%   говорят по-английски. Во сколько раз число людей, не говорящих по-французски, больше числа людей, не говорящих по-английски?
    \item У Пети в семье, помимо папы, мамы и бабушки, есть ещё братья и сёстры. Средний возраст папы, мамы и бабушки на 15 лет больше среднего возраста детей и на 10 лет больше среднего возраста всех членов семьи. Сколько в семье детей?
    \item Дед Мороз раздал детям 47 шоколадок так, что каждая девочка получила на одну шоколадку больше, чем каждый мальчик. Затем дед Мороз раздал тем же детям 74 мармеладки так, что каждый мальчик получил на одну мармеладку больше, чем каждая девочка. Сколько всего было детей?
        
    \end{enumerate}
\end{document}
