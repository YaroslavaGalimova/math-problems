\documentclass[a4paper,12pt]{article}
\usepackage[utf8]{inputenc}
\usepackage[russian]{babel}
\usepackage{amsmath}
\usepackage{amssymb}
\usepackage{enumitem}
\usepackage{graphicx}
\usepackage{hyperref}
\usepackage[a4paper, top= 0.5cm, bottom=2cm, left=2cm, right=2cm]{geometry}

\hypersetup{
    colorlinks=true,
    linkcolor=blue,
    urlcolor=blue, 
}

\title{Самый-самый}

\begin{document}
\maketitle
    \subsection*{} \href{https://kvant.mccme.ru/1988/09/pravilo_krajnego.htm}{Настоятельно рекомендую прочитать статью тут!!!}
    \subsection*{Что хотим рассматривать?} Принцип крайнего заключается в рассмотрении экстремальных (то есть самых-самых по некоторому признакому) объектов: самая большая фигура, самая крайняя точка, самый высокий дом... Осталось найти признак, по которому мы будем рассматривать экстремальный объект.
    \subsection*{Задачи}
    \begin{enumerate}
        \item Шахматная доска разбита на доминошки 1 × 2. Докажите, что
какая-то пара домино образует квадратик 2 × 2.
        \item Слониха купила для своих 7 слонят семь барабанов разных размеров и семь пар палочек разной длины. Если слонёнок видит, что у него и барабан больше, и палочки длиннее, чем у кого-то из братьев, он начинает громко барабанить. Какое наибольшее число слонят сможет начать барабанить?
        \item По кругу выписаны несколько чисел, каждое равно сумме двух соседних, поделённой на 2. Докажите, что все числа равны.
        \item Можно ли на клетчатой плоскости расставить конечное колличество коней так, чтобы каждый бил хотя бы пять других?
        \item \begin{enumerate}
            \item На прямой заданы точки так, что любая точка является серединой какого-либо отрезка между двумя другими точками. Докажите, что точек бесконечно много.
            \item А если точки располагаются на плоскости?
        \end{enumerate}
        \item На плоскости синим и красным цветом окрашено несколько точек так, что никакие три точки одного цвета не лежат на одной прямой (точек каждого цвета не менее трех). Докажите, что можно найти треугольник с одноцветными вершинами, на трех сторонах которого лежит не более двух точек другого цвета. 
        \item На полях клетчатой квадратной доски написаны числа (по одному в каждой клетке) таким образом, что каждое число является средним арифметическим  своих соседей. Докажите, что все числа на доске одинаковые.
        \item Маляр-хамелеон ходит по клетчатой доске как хромая ладья (на одну клетку по вертикали или горизонтали). Попав в очередную клетку, он либо перекрашивается в её цвет, либо перекрашивает клетку в свой цвет. Белого маляра-хамелеона кладут на чёрную доску размером 8×8 клеток. Сможет ли он раскрасить её в шахматном порядке? \\
        Подсказка на азбуке морзе, чтобы сначала попробовать решить без неё: .-. .- ... ... -- --- - .-. .. - . / -- --- -- . -. - --..-- / -.- --- --. -.. .- / -... -.-- .-.. .- / .--. . .-. . -.- .-. .- ---- . -. .- / .--. --- ... .-.. . -.. -. .-.- .-.- / -.- .-.. . - -.- .- .-.-.-
        \item Дана таблица n×n, в каждой её клетке записано число, причём все числа различны. В каждой строке отметили наименьшее число, и все отмеченные числа оказались в разных столбцах. Затем в каждом столбце отметили наименьшее число, и все отмеченные числа оказались в разных строках. Докажите, что оба раза отметили одни и те же числа.
    \end{enumerate}
\end{document}
