\documentclass[a4paper,12pt]{article}
\usepackage[utf8]{inputenc}
\usepackage[russian]{babel}
\usepackage{amsmath}
\usepackage{amssymb}
\usepackage{enumitem}
\usepackage{emoji}
\usepackage[a4paper, top=1cm, bottom=2cm, left=2cm, right=2cm]{geometry}

\title{Маленький разнобой на разные темы}
\date{}

\begin{document}
\maketitle

\subsection*{Комбинаторика}

\begin{enumerate}[label=\textbf{\arabic*.}]
    \item Боб много путешествовал. Однажды он сказал, что в некоторой стране в каждое озеро впадает 2 реки и из каждого озера вытекает 5 рек. Может ли такое быть?
    \item У Боба 10 друзей, и в течение нескольких дней он приглашает некоторых из них в гости так, что компания ни разу не повторяется (в какой-то из дней он может не приглашать никого). Сколько дней он может так делать?
\end{enumerate}

\subsection*{Геометрия}

\begin{enumerate}
    \item Биссектриса внешнего угла при вершине C треугольника ABC пересекает описанную окружность в точке D. Докажите, что AD = BD.
\end{enumerate}

\subsection*{Просто логические задачи $\ddot\smile$}

\begin{enumerate}
    \item На столе лежат в ряд пять монет: средняя  — вверх орлом, а остальные  — вверх решкой. Разрешается одновременно перевернуть три рядом лежащие монеты. Можно ли при помощи нескольких таких переворачиваний все пять монет положить вверх орлом?
    \item Обязательно ли среди двадцати пяти "медных" монет (т.е. монет достоинством 1, 2, 3, 5 коп.) найдётся семь монет одинакового достоинства?
\end{enumerate}

\end{document}
