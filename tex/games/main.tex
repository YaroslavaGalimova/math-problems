\documentclass[a4paper,12pt]{article}
\usepackage[utf8]{inputenc}
\usepackage[russian]{babel}
\usepackage{amsmath}
\usepackage{amssymb}
\usepackage{enumitem}
\usepackage{graphicx}
\usepackage{hyperref}
\usepackage[a4paper, top= 1cm, bottom=2cm, left=2cm, right=2cm]{geometry}

\hypersetup{
    colorlinks=true,
    linkcolor=blue,
    urlcolor=blue, 
}

\title{Давайте сыграем?}
\date{}

\begin{document}
\maketitle
    \subsection*{Задачи}
    \begin{enumerate}
    \subsubsection*{Задачи-шутки}
    \item Двое по очереди ставят ладей на шахматную доску так, чтобы ладьи не били друг друга. Проигрывает тот, кто не может сделать ход. Кто выиграет?
    \item Имеется три кучки камней: в первой – 10, во второй – 15, в третьей – 20. За ход разрешается разбить любую кучку на две меньшие. Проигрывает тот, кто не сможет сделать ход. Кто выиграет?
    \item Двое по очереди ломают шоколадку 6*8. За ход разрешается сделать прямолинейный разлом любого из кусков вдоль углубления. Проигрывает тот, кто не сможет сделать ход.
    \subsubsection*{Повтори, если сможешь}
    \item Двое по очереди кладут пятаки на круглый стол, причем так, чтобы они не накладывались друг на друга. Проигрывает тот, кто не может сделать ход.
    \item Имеется две кучи камней по семь камней в каждой. За ход разрешается взять любое количество камней, но только из одной кучки. Проигрывает тот, кому нечего брать.
    \item Двое играют в следующую игру. Каждый игрок по очереди вычеркивает 9 чисел (по своему выбору) из последовательности 1,2,...,100,101. После одиннадцати таких вычеркиваний останутся 2 числа. Первому игроку присуждается столько очков, какова разница между этими оставшимися числами. Доказать, что первый игрок всегда сможет набрать по крайней мере 55 очков, как бы ни играл второй.
    \subsubsection*{Выигрышные-проигрышные позиции}
    \item Ладья стоит на поле a1. За ход разрешается сдвинуть ее на любое число клеток вправо или на любое число клеток вверх. Выигрывает тот, кто поставит ладью на поле h8.
    \subsubsection*{И просто стратегии...}
    \item Белая ладья преследует чёрного слона на доске 3×1969 клеток (они ходят по очереди по обычным правилам). Как должна играть ладья, чтобы взять слона? Первый ход делают белые.
        
    \end{enumerate}
\end{document}
