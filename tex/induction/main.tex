\documentclass[a4paper,12pt]{article}
\usepackage[utf8]{inputenc}
\usepackage[russian]{babel}
\usepackage{amsmath}
\usepackage{amssymb}
\usepackage{enumitem}
\usepackage{graphicx}
\usepackage{hyperref}
\usepackage[a4paper, top= 1cm, bottom=2cm, left=2cm, right=2cm]{geometry}

\hypersetup{
    colorlinks=true,
    linkcolor=blue,
    urlcolor=blue, 
}

\title{Индукция}

\begin{document}
\maketitle
    \subsection*{} \href{chrome-extension://efaidnbmnnnibpcajpcglclefindmkaj/https://old.mccme.ru/free-books/shen/shen-induction.pdf}{Хорошая статья на эту тему!}
    \subsection*{} Индукция - это формализация слов "и так далее". Она позволяет строго доказывать задачи, в которых описан процесс и требуется доказать свойство для каждого шага процесса. Индукция всегда состоит \textbf{из базы, предположения и перехода}. Разберём на примере, как это работает...  
    \subsection*{Задачи}
    \begin{enumerate}
        \item \underline{Упражнение для осознания, которое мы сразу разберём}\\ На доске написаны сто цифр: нули и единицы (в любой комбинации). Разрешается выполнять два действия: (а) заменять первую цифру (нуль на единицу и наоборот); (б) заменять цифру, стоящую после первой единицы. Показать, что с помощью нескольких таких замен можно получить любую комбинацию из ста нулей и единиц.
        \item Денис нарисовал на плоскости треугольник. Владислав провел несколько прямых, которые разделили треугольник на части. Докажите, что хотя бы одна из этих частей снова треугольник.
        \item Доказать, что любой из квадратов 2×2, 4×4, 8×8;...$2^n$×$2^n$ из которого вырезан угловой квадратик 1 ×1, можно разрезать на уголки из трёх клеток.
        \item Докажите, что неоднозначное натуральное число больше произведения своих цифр.
        \item Докажите, что сумма нескольких подрядыдущих нечётных натуральных чисел - квадрат натурального числа.
        \item Докажите, что $2^n$ > n.
        \item На полке расставили 100 томов энциклопедии. Разрешается взять несколько подряд идущих томов и переставить их в обратном порядке. Докажите, что такими операциями можно расставить тома по порядку, независимо от того, как их расставили.
        \item На столе стоят $2^n$ стаканов с водой. Разрешается взять любые два стакана и уравнять в них количества воды, перелив часть воды из одного стакана в другой. Докажите, что с помощью таких операций можно добиться того, чтобы во всех стаканах было поровну воды.
    \end{enumerate}
\end{document}
