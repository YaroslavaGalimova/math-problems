\documentclass[a4paper,12pt]{article}
\usepackage[utf8]{inputenc}
\usepackage[russian]{babel}
\usepackage{amsmath}
\usepackage{amssymb}
\usepackage{enumitem}
\usepackage{graphicx}
\usepackage{hyperref}
\usepackage[a4paper, top= 1cm, bottom=2cm, left=2cm, right=2cm]{geometry}

\hypersetup{
    colorlinks=true,
    linkcolor=blue,
    urlcolor=blue, 
}

\title{Где логика?}
\date{}

\begin{document}
\maketitle
    \subsection*{Задачи}
    \begin{enumerate}
    \item На острове аборигенов живут рыцари и лжецы. Однажды аборигены, среди которых были как рыцари, так и лжецы, встали в хоровод, и каждый произнес: «Из двух людей, стоящих рядом со мной, один — рыцарь, а другой — лжец». Сколько среди них рыцарей, если известно, что в хороводе был 21   абориген? А могло ли их быть 20?
    \item На острове живут рыцари, которые всегда говорят правду, и лжецы, которые всегда лгут. Население острова составляет 1000 человек и сосредоточено в 10 селах (в каждом селе не менее двух человек). Однажды каждый островитянин заявил, что все его односельчане — лжецы. Сколько лжецов живет на острове?
    \item В комнате собрались 8 человек. Некоторые из них лгут, а остальные — честные люди, всегда говорящие правду. Один из собравшихся сказал: « Здесь нет ни одного честного человека». Второй сказал: «Здесь не больше одного честного человека». Третий сказал: «Здесь не более двух честных людей» и т.д. до восьмого, который сказал: «Здесь не более семи честных людей». Сколько в комнате честных людей?
    \item За круглым столом сидят 60   людей. Каждый из них либо рыцарь, который всегда говорит правду, либо лжец, который всегда лжёт. Каждый из сидящих за столом произнёс фразу: “Среди следующих 3   человек, сидящих справа от меня, не более одного рыцаря”. Сколько рыцарей могло сидеть за столом? Укажите все возможные варианты и докажите, что нет других.
    \item Один абориген Острова рыцарей и лжецов сказал другому: “Я лжец или ты рыцарь”. Можно ли по этой фразе определить, кто кем является?
    \item На острове живут 100 рыцарей, которые всегда говорят правду, и 100 лжецов, которые всегда лгут. У каждого из них есть хотя бы один друг. Однажды на острове ровно 100 человек сказали: "Все мои друзья – рыцари"и ровно 100 человек сказали: "Все мои друзья – лжецы". Найдите наименьшее возможное количество пар друзей на острове, один из которых рыцарь, а другой лжец.
    \item На чудесном острове Логики живут господа-аборигены двух племён: рыцари и лжецы. Рыцари всегда говорят правду, а лжецы всегда обманывают. Аборигенов поставили на большую доску так, что в каждой клетке доски 4× 4   стоит абориген. Какое наибольшее число из них может произнести фразу “У меня есть сосед-лжец”? Соседи считаются только по стороне.
    \item  20   островитян приехали на турнир по настольным играм. В первый день турнира все собравшиеся сели за круглый стол, и перед началом каждый заявил: “Оба моих соседа лжецы”. Во второй день один островитянин заболел, и за круглый стол сели только 19   игроков. На этот раз каждый сказал: “Если я рыцарь, то мои соседи - лжецы. Если я лжец, то мои соседи - рыцари”. Кто заболел: рыцарь или лжец?
    \item На экзамен по прорицаниям к профессору Трелони пришло 13   юных волшебников. Перед экзаменом профессор попросила каждого сделать какое-нибудь предсказание, кто сдаст экзамен. Все сделали одно и то же предсказание: “Все, кроме, возможно, меня и моих соседей, не сдадут экзамен”. В итоге оказалось, что предсказание оказалось верным в точности у тех, кто сдал экзамен. Сколько человек в тот день сдали экзамен?
    \item На некотором острове живёт 100   человек, каждый из которых является либо рыцарем, который всегда говорит правду, либо лжецом, который всегда лжёт.

Однажды все жители этого острова выстроились в ряд, и первый из них сказал:


“Количество рыцарей на этом острове является делителем числа 1"

Затем второй сказал:


“Количество рыцарей на этом острове является делителем числа 2"

И так далее до сотого, который сказал:


“Количество рыцарей на этом острове является делителем числа 100"

Определите, сколько рыцарей может проживать на этом острове. Найдите все ответы и докажите, что других нет.
    \end{enumerate}
\end{document}
