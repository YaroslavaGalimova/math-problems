\documentclass[a4paper,12pt]{article}
\usepackage[utf8]{inputenc}
\usepackage[russian]{babel}
\usepackage{amsmath}
\usepackage{amssymb}
\usepackage{enumitem}
\usepackage{graphicx}
\usepackage{hyperref}
\usepackage[a4paper, top= 1cm, bottom=2cm, left=2cm, right=2cm]{geometry}

\hypersetup{
    colorlinks=true,
    linkcolor=blue,
    urlcolor=blue, 
}

\title{Новогодний разнобой}

\begin{document}
\maketitle
    \begin{enumerate}
        \item Дед Мороз путешествует между соединёнными дорогами домами, которые расположены в виде:
        \begin{enumerate}
            \item Цикла
            \item Полного графа на 5 вершин
            \item Полного графа на 4 вершины
            \item Полного двудольного графа с долями 3 и 3
        \end{enumerate}
        Дед Мороз хочет начать в каком-нибудь одном доме, затем проехать по всем дорогам и вернуться в изначальный дом. Получится ли у него это?
        \item Гирлянда состоит из 5 фонариков, каждый может гореть одним из 3 цветов: красный, зелёный или синий. Никакой красный фонарик не может находится рядом с другим красным фонариков. Никакой синий фонарик не может быть соседом двух одноцветных фонариков. Сколько способов зажечь каждый фонарик гирлянды, чтобы выполнялись эти условия?
        \item Не все дети в этом году вели себя хорошо. Сто ребят сделали от 1 до 200 плохих поступков. Первый совершил плохой поступок только 1 раз, второй - три, третий - пять и так далее... Последний совершил 199 плохих поступков. Дед Мороз собирается принести им под ёлку столько угольков, сколько плохих поступков совершили дети. Сколько угольков ему надо раздобыть перед тем, как отправляться в путь?
        \item На новый год Саша загадал желание - выиграть в какой-нибудь олимпиаде. Известно, что всего есть 60 участников олимпиад: 35 из них выиграют олимпиаду по математике, 35 - по физике, а 10 из них выиграют и то, и другое. Правда ли, что желание Саши сбудется? 
        \item Гости сели за круглый праздничный стол на стулья, пронумерованные от 1 до 8. Известно, что стулья стоят в произвольном порядке, но каждое число делится на разность своих соседей. Известно, что стулья с номерами 2 и 5 стоят рядом. Докажите, что стулья с номерами 4 и 6 тоже стоят рядом.
        \item У Деда Мороза есть огромный сейф, внутри которого хранятся подарки. Чтобы никто не пробрался в его сейф, на нём установлена система с паролем. Пароли в системе составляются из букв английского алфавита (26 букв) и цифр (10 цифр). При этом требуется, чтобы в пароле содержались цифра и заглавная буква. Пользователь допускается в систему, если предъявленный им пароль отличается от установленного не более чем в одном символе. Сколько паролей, соответствующих требованиям составления, позволят войти в сейф, если для пользователя был установлен пароль New2025year?
        \item На новый год Саше подарили игру, и он решил попробовать  поиграть в неё с другом. Поле представляет из себя клетчатую полоску $1 \times 2025$. На неё можно ставить фишки размером $1 \times 2$, $1 \times 3$ или $1 \times 4$. Проигрывает тот, кто не может сделать ход. Саша хочет придумать стратегию игры, чтобы всегда побеждать своего товарища. Как ему это сделать и под каким номером ему нужно ходить?
        \item Кто дочитал до этого задания, тот большой молодец и у того обязательно всё получится в новом году! С наступающим праздником!
    
    \end{enumerate}
\end{document}
