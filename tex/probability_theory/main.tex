\documentclass[a4paper,12pt]{article}
\usepackage[utf8]{inputenc}
\usepackage[russian]{babel}
\usepackage{amsmath}
\usepackage{amssymb}
\usepackage{enumitem}
\usepackage{graphicx}
\usepackage{hyperref}
\usepackage[a4paper, top= 1cm, bottom=2cm, left=2cm, right=2cm]{geometry}

\hypersetup{
    colorlinks=true,
    linkcolor=blue,
    urlcolor=blue, 
}

\title{Теория вероятности}
\date{}

\begin{document}
\maketitle
    \subsection*{Введение}
    \textbf{Вероятность} — это числовая мера возможности наступления события, принимающая значения от 0 до 1. Если вероятность равна 0, событие никогда не произойдет; если 1 — событие обязательно произойдет. \\
    \textbf{Случайное событие} — это результат эксперимента, который может произойти или не произойти. Например, подбрасывание монеты (орел или решка). \\
    \textbf{Общее число исходов} — количество всех возможных результатов случайного эксперимента. Например, при подбрасывании шестигранной кости общее число исходов равно 6. \\
    \textbf{Сумма вероятностей:} вероятности всех возможных исходов случайного эксперимента в сумме равны 1. \\
    \textbf{Как посчитать вероятность события A?} Для начала нам хватит этого определения: $P(A) = \frac{A}{\Omega}$
    \subsection*{Загадки для разминки}
    \begin{enumerate}
        \item Студент очень волнуется перед экзаменом. У него есть 72 билета, из которых он не знает 10. Какова вероятность, что ему не повезёт?
        \item У студента есть две попытки сдать экзамен, он всё ещё не выучил всего 10 билетов. С какой вероятностью оба раза ему не повезёт?
        \item Студент всё же не сдал экзамен, потому что ему попался сложный билет (а может, ему надо было готовиться лучше). Правда ли, что вероятность того, что на пересдаче ему попадётся этот же билет, меньше, ведь он ему уже попадался?
        \item На этот раз студент решил основательно подготовиться к экзамену по вождению в ГИБДД. У него есть 20 теоретических билетов, из них он умеет решать целых 14! Правда ли, что если его попросят решить 2 билета абсолютно без ошибок, он скорее всего справится с экзаменом (то есть вероятность этого будет более 50\%)?
        \item Совсем расстроившись, студент пошёл играть в казино в рулетку (круглое поле с 36 секторами красного и черного цвета и 37-ым сектором зелёного цвета). Он может ставить как на красное, так и на чёрное поле, а также на единственный зелёный сектор. Объясните, почему в среднем чем больше он играет, тем больше он проигрывает и отдаёт деньги казино?
    \end{enumerate}

    \subsection*{Задачи}
    \begin{enumerate}
        \item А почему студент был так плохо готов к сессии? А потому что он целую неделю вместо подготовки смотрел фильмы про дикий запад. В одном из них три усталых ковбоя зашли в салун, и повесили свои шляпы на бизоний рог при входе. Когда глубокой ночью ковбои уходили, они были не в состоянии отличить одну шляпу от другой и поэтому разобрали три шляпы наугад. Найдите вероятность того, что никто из них не взял свою собственную шляпу.
        \item Всё же студента отчислили из университета, но зато предложили участвовать в соревнованиях по стрельбе из рогатки, пневматического пистолета и ружья. Вероятность поражения мишени из рогатки равна 0,2  , из пистолета — 0,7  , из ружья — 0,8. Студент стрелял из каждого оружия по два раза. Найти вероятность того, что он допустил только один промах.
        \item В городе, где живет Рассеянный Ученый (да-да, тот самый отчисленный студент), телефонные номера состоят из 7 цифр. Ученый легко запоминает телефонный номер, если этот номер палиндром, то есть он одинаково читается слева направо и справа налево. Например, номер 4435344 Ученый запоминает легко, потому что этот номер палиндром. А номер 3723627 не палиндром, поэтому Ученый такой номер запоминает с трудом. Найдите вероятность того, что телефонный номер нового случайного знакомого Ученый запомнит легко.
        \item Мальчик Петя любит играть в кубики. У него есть 10 кубиков с буквами: М, М, А, А, А, Т, Т, Е, И, К. Он их ставить друг на друга, чтобы получилась башня. С какой вероятностью он сможет прочитать сверху вниз слово МАТЕМАТИКА?
        \item Монетку подбросили 1001 раз. Какова вероятность того, что выпало более 500 орлов?
        \item За круглый стол в случайном порядке рассаживаются Белоснежка, злая ведьма и 5 гномов (двое охраняют мероприятие). Найдите вероятность того, что Белоснежка и злая ведьма не будут сидеть вместе.
        \item Один студент вышел в магазин за 5 минут до закрытия. На его пути есть 3 светофора. Чтобы дойти от дома до первого светофора нужна одна минута. Также между каждыми соседними светофорами нужна минута. И от последнего светофора до магазина тоже идти ровно минуту. Как вы знаете светофоры иногда горят красным, причем как назло целую минуту, и с вероятностью 30\% светофор загорится красным, когда студент к нему подойдёт. НО! В мире есть высшие силы, и если он попал на один красный светофор, то на следующий раз вероятность томительного ожидания зелёного света станет 10\%. Этот плюсик в карму работает только на ближайшем светофоре, и на нем правила повторяются. С какой вероятность студент успеет за покупками до закрытия?
    \end{enumerate}
    

    \subsection*{Парадоксы}
    \begin{enumerate}
        \item Парадокс математика Мартина Гарднера: «Предположим, вам с другом предложили два конверта, в одном из которых лежит некая сумма денег X, а в другом — сумма вдвое больше. Вы независимо друг от друга вскрываете конверты, пересчитываете деньги, после чего можете обменяться ими. Конверты одинаковые, поэтому вероятность того, что вам достанется конверт с меньшей суммой, составляет $1/2$. Допустим, вы открыли конверт и обнаружили в нем \$10. Следовательно, в конверте вашего друга может быть равновероятно \$5 или \$20. Если вы решаетесь на обмен, то можно подсчитать математическое ожидание итоговой суммы — то есть, ее среднее значение. Она составляет $1/2 * (x/2) +1/2* (x *2)=5/4 * x$. Таким образом, обмен вам выгоден. И, скорее всего, ваш друг будет рассуждать точно так же. Но очевидно, что обмен не может быть выгоден вам обоим. В чем же ошибка?»
        \item Парадокс Монти Холла из фильма "Двадцать одно": «Допустим, некому игроку предложили поучаствовать в известном американском телешоу Let’s Make a Deal, которое ведет Монти Холл, и ему необходимо выбрать одну из трех дверей. За двумя дверьми находятся козы, за одной — главный приз, автомобиль, ведущий знает расположение призов. После того, как игрок делает свой выбор, ведущий открывает одну из оставшихся дверей, за которой находится коза, и предлагает игроку изменить свое решение. Стоит ли игроку согласиться или лучше сохранить свой первоначальный выбор?»
    \end{enumerate}
\end{document}
