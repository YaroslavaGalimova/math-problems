\documentclass[a4paper,12pt]{article}
\usepackage[utf8]{inputenc}
\usepackage[russian]{babel}
\usepackage{amsmath}
\usepackage{amssymb}
\usepackage{enumitem}
\usepackage{graphicx}
\usepackage{hyperref}
\usepackage[a4paper, top= 1cm, bottom=2cm, left=2cm, right=2cm]{geometry}

\hypersetup{
    colorlinks=true,
    linkcolor=blue,
    urlcolor=blue, 
}

\title{Задачи с олимпиады "Высшая проба"}

\begin{document}
\maketitle
    \subsection*{Задачи}
    \begin{enumerate}
        \item Карлсону на день рождения подарили большую банку малинового варенья. В течение 99 дней он ел варенье по следующему правилу: для всех k = 1,2,…,99, в k-й день Карлсон ел $\frac{1}{k + 1}$ от текущего остатка(в первый день он съел половину всего варенья, во второй - треть от остатка, и т.д.). Какая часть от изначального объёма варенья осталась у Карлсона через 99 дней?
        \item Найдите наибольшее натуральное число n, равное сумме двух различных натуральных делителей числа n+15.  
        \item Известно, что число $a + \frac{1}{a}$ целое. Докажите, что число $a^2 + \frac{1}{a^2}$ тоже целое.
        \item Сколько существует чисел от 1 до 1000000, не являющихся ни полным квадратом, ни полным кубом, ни четвертой степенью?
        \item На сторонах треугольника взяты точки, делящие стороны в одном итомжеотношении(вкаком-либо одном направлении обхода). Докажите, что точки пересечения медиан данного треугольника и треугольника, имеющего вершинами точки деления, совпадают.
        \item Докажите, что существует степень тройки, оканчивающаяся на 001.
    \end{enumerate}
\end{document}
